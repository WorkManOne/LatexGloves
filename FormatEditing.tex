\documentclass{article}
\usepackage[utf8]{inputenc}
\usepackage[T2A]{fontenc}
\usepackage[english,russian]{babel}
\usepackage{indentfirst}
\usepackage[mag=1000,a4paper,   	
		left=3cm , right=3cm, top=2cm, bottom=2cm,
	]{geometry}


\begin{document}
\begin{center}
\Huge \bf{ДАНИИЛ ХАРМС}
\end{center}

\vspace{1cm}
\begin{center}
\section*{СТОЛЯР КУШАКОВ}
\end{center}
    
Жил"=был столяр. Звали его  Кушаков.

    Однажды  вышел  он из дому и пошел в лавочку, 
    купить столярного клея.

    Была  оттепель,  и  на  улице было очень
скользко.

Столяр прошел несколько шагов, поскользнулся, 
упал и расшиб себе лоб.

    "--* \textit{Эх!} "--- сказал  столяр,  встал,  пошел в
аптеку, купил пластырь и заклеил себе лоб.

    Но когда он вышел на улицу и сделал  несколько шагов,  
    он опять поскользнулся, упал
и расшиб себе нос.

"--* \textit{Фу!}  "--- сказал столяр, пошел в  аптеку,
купил пластырь и заклеил пластырем себе нос.

    Потом он опять  вышел  на  улицу,  опять
поскользнулся, упал и расшиб себе щеку.

    Пришлось опять пойти в аптеку и заклеить
пластырем щеку.

"--* \textit{Вот что,} "---  сказал столяру аптекарь. "---
\textit{Вы так часто  падаете и расшибаетесь,  что я
советую ваь купить пластырей несколько штук.}

"--* \textit{Нет,} "--- сказал столяр, "--- \textit{больше 
не упаду!}

    Но когда он вышел на улицу, то опять поскользнулся, упал и расшиб себе подбородок.

"--* \textit{Паршивая гололедица!}  "---  
закричал столяр и опять побежал в аптеку.

"--* \textit{Ну вот видите,}  "---  сказал аптекарь,  "---
\textit{Вот вы опять упали.}

"--* \textit{Нет!} "--- закричал столяр. "---  
\textit{Ничего
     слышать не хочу! Давайте скорее пластырь!}

    Аптекарь  дал пластырь;  столяр  заклеил
себе подбородок и побежал домой.

    А дома его не узнали и не пустили в квартиру.

    "--* \textit{Я столяр Кушаков!} "--- закричал столяр.

    "--* \textit{Рассказывай!} "---  отвечали из квартиры и
заперли дверь на крюк и на цепочку.

    Столяр Кушаков постоял на лестнице, 
    плюнул и пошел на улицу.
\begin{flushright}
<\dots>
\end{flushright}

\begin{center}
* * *
\end{center}

\begin{center}
\section*{СУНДУК}
\end{center}

Человек с тонкой шеей забрался в сундук,
закрыл за собой крышку и начал задыхаться.

"--* Вот,  "---  говорил,  задыхаясь человек с
тонкой шеей, "--- я задыхаюсь в сундуке, потому
что у меня тонкая шея. Крышка сундука закрыта 
и не пускает ко мне воздуха. Я буду задыхаться, 
но крышку сундука все  равно  не открою. 
Постепенно я буду умирать. Я увижу борьбу 
жизни и смерти.  Бой произойдет неестественный,  
при равных шансах, потому что естественно  
побеждает смерть, а жизнь,  обреченная на смерть,  
только тщетно  борется  с
врагом, до последней минуты не теряя напрасной 
надежды. В этой же  борьбе, которая произойдет  
сейчас,  жизнь  будет  знать способ
своей победы: для этого жизни надо заставить
мои руки открыть крышку сундука.  Посмотрим:
кто кого? Только вот ужасно пахнет  нафталином. 
Если победит жизнь, я буду вещи в  сундуке 
пересыпать  махоркой\dots  Вот  началось:
я больше  не могу дышать. Я погиб, это ясно!
Мне уже нет спасения! И ничего  возвышенного
нет в моей голове. Я задыхаюсь!..

    Ой! Что же это такое? Сейчас что"=то произошло, 
    но я не могу понять, что  именно.  Я
что"=то видел или что"=то слышал\dots

    Ой! Опять  что"=то  произошло?  Боже мой!
Мне нечем дышать. Я, кажется, умираю\dots

    А это еще что такое?  Почему я пою?  Кажется, 
    у меня болит шея\dots Но где же сундук?
Почему  я  вижу все,  что находится у меня в
комнате? Да никак  я лежу на полу!  А где же
сундук?

    Человек с тонкой шеей поднялся с пола  и
посмотрел кругом. Сундука нигде не было.  На
стульях  и  кровати лежали вещи,  вынутые из
сундука, а сундука нигде не было.

    Человек с тонкой шеей сказал:

    "--* Значит, жизнь  победила  смерть  
    неизвестным для меня способом.

    (В  черновике приписка:  жизнь  победила
смерть, где именительный падеж,  а где винительный).
\begin{flushright}
30 января 1937 г.
\end{flushright}
                        

\begin{center}
    * * *
\end{center}

\begin{center}					
\section*{}			
\end{center}	

\begin{flushleft}
\textsc{Востряков} смотрит в окно на улицу:
\end{flushleft}

   Смотрю в окно и вижу снег.

   Картина зимняя давно душе знакома.

   Какой"=то глупый человек

   Стоит в подъезде противоположного дома.

   Он держит пачку книг под мышкой

   Он курит трубку с медной крышкой.

   Теперь он быстрыми шагами

   Дорогу переходит вдруг,

   Вот он исчез в оконной раме.

   \begin{center}
    (Стук в дверь).
   \end{center}
   
   Теперь я слышу в двери стук.

   Кто там?
\begin{flushleft}
    \textsc{Голос за дверью:}
\end{flushleft}

   Откройте. Телеграмма.

\begin{flushleft}
    \textsc{Востряков:}
\end{flushleft}

   Врет. Чувствую, что это ложь.

   И вовсе там не телеграмма.

   Я сердцем чую острый нож.

   Открыть иль не открыть?

\begin{flushleft}
    \textsc{Голос за дверью:}
\end{flushleft}

   Откройте!

   Чего вы медлите?

\begin{flushleft}
    \textsc{Востряков:} 
\end{flushleft}

   Постойте!

   Вы суньте мне под дверь посланье.

   Замок поломан. До свиданья.
\begin{flushleft}
    \textsc{Голос за дверью:}
\end{flushleft}

   Вам нужно в книге расписаться.

   Откройте мне скорее дверь.

   Меня вам нечего бояться,

   Скорей откройте. Я не зверь.

\begin{flushleft}
    \textsc{Востряков} (приоткрывая дверь):
\end{flushleft}

   Войдите. Где вы? Что такое?

\begin{center}
    (Смотрит за дверь).
\end{center}
               
   Куда жеон пропал?  Он не мог далеко уйти.

   Спрятаться  тут негде.  Куда же он делся?

   Улица совсем пустая. Боже мой! И на снегу
   нет следов! Значит, никто к моей двери не
   подходил.  Кто же стучал?  Кто говорил со
   мной через дверь?

\begin{center}
    (Закрывает дверь).
   \end{center}

\begin{flushright}
    1937 -- 1938 гг.
\end{flushright}
              
\begin{center}
    * * *
\end{center}	
\begin{center}
\section*{СЛУЧАЙ С ПЕТРАКОВЫМ}
\end{center}

    Вот  однажды  Петраков хотел спать лечь,
да лег мимо кровати. Так он об пол ударился,
что лежит на полу и встать не может.

    Вот  Петраков собрал  последние  силы  и
встал на четвереньки. А силы его покинули, и
он опять упал на живот и лежит.

    Лежал Петраков на полу часов пять.  
    Сначала просто так лежал, а потом заснул.

    Сон  подкрепил  силы  Петракова. 
    Он проснулся совершенно здоровым, встал,  прошелся
по комнате и  лег осторожно на кровать. "<Ну,
"--- думает, "--- теперь посплю">.  А спать"=то  уже
и не хочется.  Ворочается Петраков с боку на
бок и никак заснуть не может.

    Вот, собственно, и все.
\begin{flushright}
<\dots>
\end{flushright}		  
\end{document}