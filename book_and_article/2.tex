Однажды Орлов объелся толченым горохом и
умер. А Крылов, узнав об этом, тоже умер. А
Спиридонов умер сам собой. А жена Спиридонова 
упала с буфета и тоже умерла. А дети Спиридонова
утонули в пруду. А бабушка Спиридонова 
спилась и пошла по дорогам. А Михайлов
перестал причесываться и заболел паршой. А
Круглов нарисовал даму с кнутом и сошел с
ума. А Перехрестов получил телеграфом четыреста
 рублей и так заважничал, что его вы
толкали со службы.
    
Хорошие люди не умеют поставить себя на
твердую ногу.
\begin{flushright}
    22 августа 1936 года.
\end{flushright}