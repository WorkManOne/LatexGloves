Удивительный случай случился со мной: я
вдруг забыл, что идет раньше "--- 7 или 8.
    
Я отправился к соседям и спросил их, что
они думают по этому поводу.
    
Каково же было их и мое удивление, когда
они вдруг обнаружили, что тоже не могут
вспомнить порядок счета. 1,2,3,4,5 и 6 
помнят, а дальше забыли.
    
Мы все пошли в комерческий магазин "Гастроном", 
что на углу Знаменской и Бассейной
улицы, и спросили кассиршу о нашем недоумении. 
Кассирша грустно улыбнулась, вынула изо
рта маленький молоточек и, слегка подвигав
носом, сказала:
    
"--* По"=моему, семь идет после восьми в том
случае, когда восемь идет после семи.
    
Мы поблагодарили кассиршу и с радостью
выбежали из магазина. Но тут, вдумываясь в
слова кассирши, мы опять приуныли, так как
ее слова показались нам лишенными всякого
смысла.
    
Что нам было делать? Мы пошли в Летний
сад и стали там считать деревья. Но дойдя в
счете до 6"=ти, мы остановились и начали спорить: 
по мнению одних дальше следовало 7, по
мнению других "--- 8.
    
Мы спорили бы очень долго, но, по счастию 
тут со скамейки свалился какой"=то ребенок 
и сломал себе обе челюсти. Это отвлекло
нас от нашего спора.
    
А потом мы разошлись по домам.
\begin{flushright}
    12 ноября 1935.
\end{flushright}               