Г о г о л ь (падает из"=за кулис на сцену 
и смирно лежит).
    
П у ш к и н (выходит, спотыкается об 
Гоголя и падает): Вот черт! Никак об Гоголя!
    
Г о г о л ь (поднимаясь): Мерзопакость
какая! Отдохнуть не дадут! (Идет, спотыкается 
об Пушкина и падает). Никак об Пушкина
спотыкнулся!
    
П у ш к и н (поднимаясь): Ни минуты покоя! 
(Идет, спотыкается об Гоголя и падает).
Вот черт! Никак опять об Гоголя!
    
Г о г о л ь (поднимаясь): Вечно во всем
помеха! (Идет, спотыкается об Пушкина и 
падает). Вот мерзопакость! Опять об Пушкина!
    
П у ш к и н (поднимаясь): Хулиганство!
Сплошное хулиганство! (Идет, спотыкается об
Гоголя и падает). Вот черт! Опять об Гоголя!
    
Г о г о л ь (поднимаясь): Это издевательство 
сплошное! (Идет, спотыкается об 
Пушкина и падает). Опять об Пушкина!
    
П у ш к и н (поднимаясь): Вот черт! 
Истинно что черт! (Идет, спотыкается об Гоголя
и падает). Об Гоголя!
    
Г о г о л ь (поднимаясь): Мерзопакость!
(Идет, спотыкается об Пушкина и падает). Об
Пушкина!
    
П у ш к и н (поднимаясь): Вот черт!
(Идет, спотыкается об Гоголя и падает за 
кулисы). Об Гоголя!
    
Г о г о л ь (поднимаясь): Мерзопакость!
(Уходит за кулисы).
\begin{center}
За сценой слышен голос Гоголя:
               
<<Об Пушкина!>>
                 
Занавес.
\end{center}

\begin{flushright}
    1934
\end{flushright}